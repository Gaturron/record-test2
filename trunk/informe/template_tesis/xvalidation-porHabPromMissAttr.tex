% Cada hablante que tenga un atributo '?' cambiarlo por el promedio de ese atributo en las demás audios
\subsection{Promediando los atributos de cada hablante sólo si es desconocido}

En el anterior esquema vimos que promediando los atributo evitamos tener valores desconocidos. Sí bien esto es cierto, no es necesario promediar en todos los casos. 

Supongamos que tenemos el mismo conjunto de datos a la Tabla \ref{datos_orig}. Si para cada hablante promediamos sus atributos estaríamos perdiendo información. El hablante 1 tiene en el Audio1 el atributo A1 definido como 1, mientras que en el Audio3 como 2. Si realizáramos el promedio, estos valores específicos para estos audios los perderíamos.

%\begin{table}[H]
%	\centering
%	\begin{tabular}{|l|l|ccccc|}
%		\hline
%		\multicolumn{2}{|l|}{Atributos} & A1 & A2 & A3 & ... & AN \\
%		\hline 
%		\textbf{Hablante 1} & \textbf{Audio1} & 1 & ? & 2 & & 2\\
%		& \textbf{Audio2} & ? & ? & 1 & ... & ? \\
%		& \textbf{Audio3} & 2 & ? & 3 & & ? \\
%		\hline
%		\textbf{Hablante 2} & \textbf{Audio1} & 1 & ? & ? & ... & ? \\
%		& \textbf{Audio2} & 1 & 2 & ? & & ? \\
%		\hline
%	\end{tabular}
%	\caption{Atributos original}
%	\label{attr_orig}
%\end{table}

Es por ello que proponemos esta variante. Cuando haya un atributo desconocido en un audio, vamos a promediarlo con los atributos de los demás audios del mismo hablante. En la tabla \ref{attr_mod} se puede ver el resultado de esta variante marcado con negrita los nuevos valores. 

Por ejemplo: para el Hablante 1, el Audio1 no tiene definido el atributo A1. Entonces vamos a promediarlo con los demás audios. El Audio1 y Audio3 sí tienen este atributo definido y sus valores son 2 y 1 respectivamente. Realizamos el promedio nuevamente: $ 2 + 1 / 2 = 1,5$ entonces el valor del atributo 1 para el Audio2 es 1,5. De esta forma, no perdemos información con respecto al audio extraído.

\begin{table}[H]
	\centering
	\begin{tabular}{|l|l|ccccc|}
		\hline
		\multicolumn{2}{|l|}{Atributos} & A1 & A2 & A3 & ... & AN \\
		\hline 
		\textbf{Hablante 1} & \textbf{Audio1} & 1 & ? & 2 & & 2\\
		& \textbf{Audio2} & \textbf{1.5} & ? & 1 & ... & \textbf{2} \\
		& \textbf{Audio3} & 2 & ? & 3 & & \textbf{2} \\
		\hline
		\textbf{Hablante 2} & \textbf{Audio1} & 1 & \textbf{2} & ? & ... & ? \\
		& \textbf{Audio2} & 1 & 2 & ? & & ? \\
		\hline
	\end{tabular}
	\caption{Atributos modificados 2}
	\label{attr_mod}
\end{table}

Podemos observar los resultados del promedio de los folds para cada clasificador en la tabla \ref{class_corr_en_pct}. En este caso Zero rule dió el porcentaje esperado y los demás por arriba de su valor. Esto nos muestra que utilizando estos atributos se puede superar el baseline.

\begin{table}[H]
	\centering
	\begin{tabular}{|l|c|c|c|c|c|c|}
		\hline
		\textbf{}  & \textbf{ZeroR} & \textbf{JRip} & \textbf{J48} & \textbf{Function SMO} & \textbf{NaiveBayes} \\ \hline
		\textbf{Promedio} & 50 & 72.44 & 73.48 & 77.19 & 74.62 \\ \hline
	\end{tabular}
	\caption{Clasificación correcta en porcentaje}
	\label{class_corr_en_pct}
\end{table}

\subsubsection{Wilcoxon y Test t de Student}

Corrimos los test estadísticos y obtuvimos que los resultados de los folds de NaiveBayes y SVM tienen evidencia estadística para ser mejor que Zero rules. 

\begin{table}[H]
	\centering
	\begin{tabular}{|l|c|c|c|c|c|c|}
		\hline
		\textbf{}  & \textbf{Student Test} & \textbf{Wilcoxon Test} \\ \hline
		\textbf{ZeroR y Ripper}  & 0.06537 & 0.1284 \\ \hline
		\textbf{ZeroR y C4.5}  & 0.06156 &  0.1111 \\ \hline
		\textbf{ZeroR y NaiveBayes}  & 0.03916 & 0.06111 \\ \hline
		\textbf{ZeroR y SVM}  &  0.02936 & 0.03522 \\ \hline
	\end{tabular}
	\caption{Resultados de cada test representado en p-valor}
	\label{res_tests_wilcoxon_student}
\end{table}

\subsection{Características del modelo de test}

Con esta esquema mejoramos las matrices de confusión. La tabla \ref{mat_conf_f1} corresponde al primer fold. Notamos que ahora sí se analiza cada audio y se intenta clasificarlo. 

\begin{table}[H]
	\centering
	\begin{tabular}{|c|c|c|}
		\hline
		Buenos Aires & Córdoba & \\ \hline
		33 & 1 & Buenos Aires\\ \hline
		0 & 0 & Córdoba\\ \hline
	\end{tabular}
	\caption{Matriz de confusión fold 1}
	\label{mat_conf_f1}
\end{table}

% hablar de los problemas encontrados y cómo mejorarlos con el próximo cross-validation 