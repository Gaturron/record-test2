%\begin{center}
%\large \bf \runtitulo
%\end{center}
%\vspace{1cm}
\chapter*{\runtitulo}

\noindent 

El uso de la lengua siempre ha caracterizado a las personas que la utilizan. La forma como nos comunicamos no sólo posee la información del mensaje a transmitir, sino que también posee características del hablante. Estas características pueden describir al hablante de distintas formas. Algunas de ellas pueden ser: su cultura, su nivel socioeconómico y su región, entre otras. En particular, Argentina no es la excepción. Nuestro país posee una fuerte componente dialéctica en su habla. Esto quiere decir que podemos saber de qué lugar proviene el hablante analizando su tonada. Hay varias regiones con esta característica definidas en el país. Nos enfocaremos en distinguir diferencias entre las regiones de Córdoba y Buenos Aires. En el presente trabajo desarrollamos un sistema de grabación a través de Internet que nos permitió recolectar grabaciones de hablantes provenientes de ambos grupos. Con este sistema de grabación, diseñamos un experimento que nos ayudó a comparar el habla de cada grupo haciendo hincapié en sus diferencias. Analizamos las dificultades técnicas que surgieron y cómo impactaron en el estudio final. Utilizando estos datos, analizamos efectivamente cuáles son las características más predominantes y cómo repercuten para una buena clasificación. Extrayendo estos atributos, utilizamos algoritmos de Machine Learning para la clasificación de hablantes en los dos grupos. 

%En el presente trabajo realizamos un experimento que consiste en comparar el habla de cada grupo para luego poder clasificar a los hablantes según a qué grupo pertenecen. Para ello, desarrollamos un sistema de grabación a través de Internet que nos permitió recolectar grabaciones de hablantes provenientes de  ambos grupos. También diseñamos en qué consiste el experimento haciendo hincapié en las diferencias del habla de cada grupo. Analizamos las dificultades técnicas que surgieron y cómo impactan en el estudio final.


%Es decir, clasificar a cada hablante según a qué grupo pertenece utilizando sus atributos. Definimos un baseline que luego lo tratamos de superar con distintos clasificadores. Para la evaluación de cada clasificador, diseñamos un modelo basado en Cross-Validation que tiene particularidades del problema propuesto. Utilizando estos resultados, evaluamos si el porcentaje de mejora de cada clasificación es estadísticamente significativo. Por último, analizamos cuáles son los atributos que nos proveen mayor ganancia de información a la hora de clasificar entre los dos grupos.

%Realizaremos un experimento donde compararemos el habla de cada grupo. Utilizando estos datos analizaremos efectivamente cuáles son las características más predominantes y cómo repercute esas diferencias en el habla. Por último, mostraremos distintos clasificadores para determinar de qué grupo proviene una grabación, analizaremos las atributos más importantes y evaluaremos la solución propuesta. 

\bigskip
