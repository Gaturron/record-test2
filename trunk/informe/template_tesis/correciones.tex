Correciones Fernando Schapachnik:

Capítulo 1 - Introducción :

Pág. 1:
TODO: Asumo que dice 'Grupo social' de costado. Igual no entiendo qué arreglar.
TODO: "El objetivo de esta tesis es realizar un sistema que pueda facilitar (soluciones) a estos problemas" => asumo que es eso
Lo demás OK. 

Pág. 2:
TODO: El experimento que tomamos como caso de estudio es (medir) las diferencias en el habla entre Córdoba y Buenos Aires. => asumo que es eso

Pág. 3:
TODO: "...en la aspiración de la /s/." => asumo que dice "Dónde" así que expliqué que se refiere al terminar una palabra.
Lo demás OK.

Capítulo 2: Diseño del experimento :

Pág. 5: 
cuáles ? OK cambiado 
"El acento se potencia cuando se realiza habla espontánea" nose que quiso marcar ahí.
"para hablantes de Córdoba esta duración es más corta (larga) que para hablantes de Buenos Aires" => OK
"Este esquema se llama AMPER \cite{amper} y lo veremos en detalle más adelante (referencia)." => cambiado por "... lo veremos más en detalle en la sección 2.1.1"
"Utiliza (?) este esquema para cubrir todo tipo de acentuación." => "Utilizamos..."


Pág. 6:
TODO: "Este trabajo estudió los acentos del español argentino utilizando todas sus combinaciones." => Releer bien el trabajo de amper pero lo puedo cambiar por "Este trabajo estudió los acentos del español argentino utilizando un esquema de frase fija y intercambiando palabras para analizar todos sus casos" 

"Para el esquema AMPER se fija un patrón de estructura de frases y se va cambiando las palabras que utiliza (marcado acá). " => cambiado por "Para el esquema AMPER se fija una estructura para la frase y se va cambiando las palabras que utiliza. " 

"En este ejemplo podemos analizar la sílaba anterior a la acentuada de estos dos grupos.(grupos BsAs Cba)" => "En este ejemplo podemos analizar la sílaba anterior a la acentuada de los dos grupos estudiados, Córdoba y Buenos Aires." 


