% Juntar por cada hablante sus audios en una muestra sola promediando sus atributos
\subsection{Promediando los atributos de cada hablante}

8 hablantes Buenos Aires. 8 Córdoba.

Cross validation similar al anterior pero con sólo 16 hablantes.

\begin{center}
	\mycirc[blue] Hablante para train \mycirc[red] Hablante para test
\end{center}

\begin{table}[H]
	\centering
	\begin{tabular}{cccccccccccc}
		& \multicolumn{11}{c}{\textit{Número de hablante}} \\
		& 1 & 2 & 3 & 4 & 5 & 6 & 7 & ... & 25 & 26 & 27 \\
		\hline \\
		Fold 1 &\mycirc[red] & \mycirc[blue] & \mycirc[blue]  & \mycirc[blue]  & \mycirc[blue]  & \mycirc[blue]  & \mycirc[blue] & ... & \mycirc[blue] & \mycirc[blue] & \mycirc[blue]  \\
		
		Fold 2 &\mycirc[blue] & \mycirc[red] & \mycirc[blue]  & \mycirc[blue]  & \mycirc[blue]  & \mycirc[blue]  & \mycirc[blue] & ... & \mycirc[blue] & \mycirc[blue] & \mycirc[blue]  \\
		
		Fold 3 &\mycirc[blue] & \mycirc[blue] & \mycirc[red]  & \mycirc[blue]  & \mycirc[blue]  & \mycirc[blue]  & \mycirc[blue] & ... & \mycirc[blue] & \mycirc[blue] & \mycirc[blue]  \\
		
		\multicolumn{11}{c}{\textit{...}}	\\
		
		Fold 16 &\mycirc[blue] & \mycirc[blue] & \mycirc[blue]  & \mycirc[blue]  & \mycirc[blue]  & \mycirc[blue]  & \mycirc[blue] & ... & \mycirc[blue] & \mycirc[blue] & \mycirc[red]   \\
		
	\end{tabular}
	\caption{Esquema de cross-validation}
	\label{}
\end{table}


Juntamos los atributos de cada hablante de la siguiente forma.

\begin{table}[H]
	\centering
	\begin{tabular}{|l|l|ccccc|}
		\hline
		\multicolumn{2}{|l|}{Atributos} & A1 & A2 & A3 & ... & AN \\
		\hline 
		\textbf{Hablante 1} & \textbf{Audio1} & 1 & ? & 2 & & 2\\
		& \textbf{Audio2} & ? & ? & 1 & ... & ? \\
		& \textbf{Audio3} & 2 & ? & 3 & & ? \\
		\hline
		\textbf{Hablante 2} & \textbf{Audio1} & 1 & ? & ? & ... & ? \\
		& \textbf{Audio2} & 1 & 2 & ? & & ? \\
		\hline
	\end{tabular}
	\caption{}
	\label{}
\end{table}

esto pasaría a:

\begin{table}[H]
	\centering
	\begin{tabular}{|l|l|ccccc|}
		\hline
		\multicolumn{2}{|l|}{Atributos} & A1 & A2 & A3 & ... & AN \\
		\hline 
		\textbf{Hablante 1} & \textbf{Audio1} & \textbf{1.5} & \textbf{?} & \textbf{1.667} & ... & \textbf{2}\\
		\hline
		\textbf{Hablante 2} & \textbf{Audio1} & \textbf{1} & \textbf{2} & \textbf{?} & ... & \textbf{?} \\
		\hline
	\end{tabular}
	\caption{}
	\label{}
\end{table}

O sea, una fila de atributos por cada hablante.

\subsection{Resultados}

\begin{table}[H]
	\centering
	\begin{tabular}{|l|c|c|c|c|c|c|}
		\hline
		\textbf{}  & \textbf{ZeroR} & \textbf{JRip} & \textbf{J48} & \textbf{Function SMO} & \textbf{NaiveBayes} \\ \hline
		\textbf{Fold 1}  &  &  &  &  &  \\ \hline
		\hline \hline
		\textbf{Promedio} &  &  &  &  &  \\ \hline
	\end{tabular}
	\caption{Clasificación correcta en porcentaje}
	\label{class_corr_en_pct}
\end{table}

\subsubsection{Wilcoxon y Test t de Student}

\begin{table}[H]
	\centering
	\begin{tabular}{|l|c|c|c|c|c|c|}
		\hline
		\textbf{}  & \textbf{Student Test} & \textbf{Wilcoxon Test} \\ \hline
		\textbf{ZeroR y JRip}  &  &  \\ \hline
		\textbf{ZeroR y J48}  &  &  \\ \hline
		\textbf{ZeroR y NaiveBayes}  &  &  \\ \hline
		\textbf{ZeroR y Function SMO}  &  & \\ \hline
	\end{tabular}
	\caption{Resultados de cada test representado en p-valor}
	\label{res_tests_wilcoxon_student}
\end{table}

% hablar de los problemas encontrados y cómo mejorarlos con el próximo cross-validation 