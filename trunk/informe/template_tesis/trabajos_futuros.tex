\chapter{Trabajos futuros}

Algunos trabajos futuros que se desprenden de este trabajo son:

\paragraph*{Filtro por ruido con SoX:} Se podría analizar mejor los audios si se le aplicara algún filtro que extraiga el ruido. De esta forma se podría mejorar los atributos acústicos. 

\paragraph*{Chequeador cruzado con grabación:} Que no se muestre la frase a decir, sino que se escuche un audio y se tenga repetirlo. Este audio es el de un hablante anterior pero que se le aplicó un filtro para evitar exponer su acento. De esta forma, chequeamos que se diga lo que se quiere decir y nos aseguramos espontaneidad.

\paragraph*{Chequeador cruzado:} Que el hablante diga si una grabación de un hablante anterior dijo lo que tenía que decir, o se equivocó. En este caso, serviría mucho para chequear los datos. 

\paragraph*{Validación de calidad de sonido:} Cuando un hablante graba un audio este debe pasar una validación en el servidor para poder continuar con el siguiente. Sino fue satisfactorio, se pude grabar devuelta. Evitaría el ruido generado por el ambiente. 

\paragraph*{Clasificación en vivo:} Realizar la clasificación al agregar un hablante nuevo. De esta forma, se puede darle una respuesta a que grupo pertenece. Revalidar el modelo al agregarse un nuevo Hablante. 
