
\chapter{Introducción}

El uso de la lengua siempre ha caracterizado a las personas que la utilizan. La forma en que nos comunicamos no sólo posee la información del mensaje a trasmitir, sino también características del hablante. Estudiar estas características del habla nos permite conocer mejor la cultura de las personas. Nos permite, por ejemplo, identificar a los hablantes para saber el lugar donde pertenecen.

Identificar y extraer características del habla son tareas muy difíciles de realizar. No sólo se deben obtener muestras muy variadas de muchos hablantes en distintas regiones, sino que también hay que prestarle importante atención a su edad, su sexo, su situación económica, grupo social, etc. Realizar un estudio de estas características es muy complejo y sobre todo, costoso. Además de estudiar cada grupo se deben utilizar muchos recursos: por ejemplo, se debe utilizar soporte para grabar en buena calidad las muestras, varios viajes para buscar los diferentes hablantes, analizar cada uno de las grabaciones de manera individual, entre otras cosas. 

\begin{figure}[h!]
	\centering
    \includegraphics[width=0.5\textwidth]{Dialectos_del_idioma_espanol_en_Argentina} 
    \caption{Dialectos del idioma español en Argentina}
    \label{fig11}
\end{figure}

El objetivo de esta tesis es realizar un sistema que pueda facilitar la resolución de estos problemas. Vamos a enfrentar cada uno de ellos e intentar resolverlos de forma computacional. De los problemas descriptos el principal radica en obtener cada grabación. Si los grupos se encuentran muy alejados esto puede ser muy costoso por los viajes. También estas grabaciones deben ser de calidad aceptable como para realizar el estudio en cuestión. Se podría utilizar el teléfono para algunos experimentos pero hay que tener en cuenta que este posee calidad de audio muy baja. De hecho, se utiliza en algunos experimentos donde esta característica no es un inconveniente. 

Es por esto que se desarrolló un sistema de grabación basado en Internet como herramienta para obtener muestras. De esta forma, se pueden realizar varias grabaciones sin necesidad de viajar a cada lugar. Es cierto que no todos los lugares poseen acceso a Internet y, si se realizara un experimento de estas características en lugares que no posean conexión, este sistema no sería útil. De cualquier forma, pensamos que su utilización soluciona muchos inconvenientes. Otra característica radica en que se puede manejar la calidad de la grabación. Utilizando distintas tecnologías a través de esta red se puede configurar la calidad para que sea lo más precisa posible para el experimento. El sistema desarrollado va a mejorar la forma de recolectar muestras y lo utilizamos en un experimento para corroborar las ventajas y desventajas del mismo.

El objetivo del sistema es obtener muestras de habla para su posterior análisis o utilización en sistemas de procesamiento de voz. El experimento que tomamos como caso de estudio es medir las diferencias en el habla entre Córdoba y Buenos Aires. El primero de estos grupos se encuentra en la zona central de nuestro país mientras que el segundo cerca del Río de la Plata, como se puede observar en la figura \ref{fig11}. En la literatura existen estudios que explican estas diferencias, por ejemplo \textit{El español en la Argentina} \cite{Fontanella2000} de Beatriz Fontanella de Weinberg y \textit{Español en la Argentina} \cite{Vidal1964} de Elena Vidal de Battini. 

Fontanella de Weinberg recopila varios trabajos de colegas que analizan el español de cada región de Argentina. Cada región se describe en un capítulo distinto y entre ellas se encuentra uno para Buenos Aires y otro para Córdoba. Por ejemplo, en la descripción de estos capítulos se señalan las siguientes diferencias para Córdoba: un sonido /r/ más suave y corto, el cambio de sonidos de /y/ y /ll/ a /j/, y la aspiración de la /s/ al terminar una palabra. También describe el estiramiento de la sílaba anterior a la acentuada en cada palabra como distintivo del acento cordobés. Por su parte, Vidal de Battini analiza región por región el uso de los fonemas importantes. Destaca la diferencia entre las dos regiones en el uso de la /r/, de la /s/ y de la /ll/, también referencia a la pronunciación de la /s/.
%pagina 85 Fontanella 

Extrayendo el análisis de estos libros se puede definir las reglas que describen a cada grupo. Las reglas son: 

\begin{itemize}

\item \textbf{Regla 1: Los hablantes de Córdoba estiran la sílaba anterior a la acentuada mientras los de Buenos Aires no lo hacen.} Cada palabra posee una sílaba con su acento primario. Para cumplir esta regla se debe estirar la sílaba anterior a esta. Si la sílaba acentuada es la primera de la palabra, entonces no se estira. Ejemplo: `Espectacular' posee su sílaba acentuada en `-lar'. La sílaba anterior, o sea `-cu-' se alarga solamente para hablantes de Córdoba. 

\item \textbf{Regla 2: Los hablantes de Córdoba aspiran y elisionan la /s/ al finalizar una palabra. Esto no sucede en Buenos Aires.} Para las palabras terminadas en /s/ se acorta la duración de éste último fonema en hablantes de Córdoba. Ejemplo: `Pájaros' posee el fonema /s/ al final. Utilizando la dialéctica de Córdoba, la /s/ final sería más suave que una de Buenos Aires. 

\item \textbf{Regla 3: Para hablantes de Córdoba, la /s/ antes de la /c/ o /t/ suenan más suaves que para hablantes de Buenos Aires.} El fonema /s/, que precede a /c/ o /t/, suena más suave en cordobeces que en porteños. Ejemplo: `Mosca' en la variante de Córdoba posee el fonema /s/ más suave que en Buenos Aires.

\item \textbf{Regla 4: La `c' antes de la `t' se pronuncia con menor frecuencia para hablantes de Córdoba que para hablantes de Buenos Aires.} El fonema /c/, que precede a /t/, no se debe pronunciar en el dialecto cordobés. Ejemplo: `Doctor' no debe sonar el fonema /c/.

\item \textbf{Regla 5: Para hablantes cordobeces la `y’ y `ll’ se pasa a `i’. No sucede esto para Buenos Aires.} Palabras con el fonema /y/ o /ll/ se pronuncian /j/. Ejemplo: `lluvia' se debe pronunciar utilizando el fonema /j/. 

\item \textbf{Regla 6: En hablantes cordobeces la /r/ no vibra mientras que en Buenos Aires pasa lo contrario.} Palabras con el fonema /r/ deben ser suaves y no vibrar en el dialecto cordobés. Ejemplo: `Espárrago' en su fonema /r/ debe ser suave en comparación con Buenos Aires. 

\end{itemize}

Normalmente estas reglas se producen en el habla espontánea y raramente en habla leída. Algunas pueden agudizarse si se encuentran en lugares económicamente más vulnerables, pero en cualquier ambiente se cumplen.

En el próximo capítulo describiremos el diseño del experimento. Éste tiene como objetivo reconocer las diferencias planteadas con las reglas mediante la grabación de frases. Estas frases fueron grabadas tanto por hablantes de Córdoba como de Buenos Aires. También describiremos cuáles frases utilizamos y el medio empleado para grabar.
