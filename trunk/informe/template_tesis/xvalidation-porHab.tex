% Un hablante para test, lo demás para train
\subsection{Un hablante para test y los demás para train}

\usetikzlibrary{shapes.geometric}

\tikzset{myshade/.style={minimum size=.4cm,shading=radial,inner color=white,outer color={#1!90!gray}}}
\newcommand\mycirc[1][]{\tikz\node[circle,myshade=#1]{};}

27 hablantes

\begin{itemize}
	\item[] \mycirc[blue] Hablante para train
	\item[] \mycirc[red] Hablante para test
\end{itemize}

\begin{tabular}{rc}
	& \textit{Hablantes} \\
	Fold 1 &\mycirc[red]\foreach \n in {1,...,6}{\mycirc[blue]} ... \foreach \n in {1,...,3}{\mycirc[blue]} \\
	
	Fold 2 &\mycirc[blue]\mycirc[red]\foreach \n in {1,...,5}{\mycirc[blue]} ... \foreach \n in {1,...,3}{\mycirc[blue]}  \\
	
	Fold 3 &\mycirc[blue]\mycirc[blue]\mycirc[red]\foreach \n in {1,...,4}{\mycirc[blue]} ... \foreach \n in {1,...,3}{\mycirc[blue]}  \\
	                         ... \\
	
	Fold 27 &\foreach \n in {1,...,7}{\mycirc[blue]} ... \foreach \n in {2,...,3}{\mycirc[blue]}\mycirc[red] \\
\end{tabular}

\subsection{Resultados}

\begin{table}[H]
	\centering
	\begin{tabular}{|l|c|c|c|c|c|c|}
		\hline
		\textbf{}  & \textbf{ZeroR} & \textbf{JRip} & \textbf{J48} & \textbf{Function SMO} & \textbf{NaiveBayes} \\ \hline
		\textbf{Fold 1}  &  &  &  &  &  \\ \hline
		\hline \hline
		\textbf{Promedio} &  &  &  &  &  \\ \hline
	\end{tabular}
	\caption{Clasificación correcta en porcentaje}
	\label{class_corr_en_pct}
\end{table}


\subsubsection{Wilcoxon y Test t de Student}

\begin{table}[H]
	\centering
	\begin{tabular}{|l|c|c|c|c|c|c|}
		\hline
		\textbf{}  & \textbf{Student Test} & \textbf{Wilcoxon Test} \\ \hline
		\textbf{ZeroR y JRip}  &  &  \\ \hline
		\textbf{ZeroR y J48}  &  &  \\ \hline
		\textbf{ZeroR y NaiveBayes}  &  &  \\ \hline
		\textbf{ZeroR y Function SMO}  &  & \\ \hline
	\end{tabular}
	\caption{Resultados de cada test representado en p-valor}
	\label{res_tests_wilcoxon_student}
\end{table}

% hablar de los problemas encontrados y cómo mejorarlos con el próximo cross-validation 

Problemas:
\begin{itemize}
	\item desbalance
	\item missing values
\end{itemize}