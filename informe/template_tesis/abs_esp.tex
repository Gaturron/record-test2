%\begin{center}
%\large \bf \runtitulo
%\end{center}
%\vspace{1cm}
\chapter*{\runtitulo}

\noindent 

El uso de la lengua siempre ha caracterizado a las personas que la utilizan. La forma en como nos comunicamos no sólo posee la información del mensaje a trasmitir, sino que también posee características del hablante. Estas características pueden describir al hablante de distintas formas. Algunas de ellas pueden ser: su cultura, su economía, su región entre otras. 

Particularmente en Argentina no es la excepción. Nuestro país posee una fuerte componente dialéctica en su habla. Esto quiere decir que podemos saber de que lugar proviene el hablante analizando su tonada. Hay varias regiones definidas a través del país. En este trabajo nos enfocaremos en distinguir diferencias entre la región de Córdoba y Buenos Aires. Realizaremos un experimento donde compararemos el habla de cada grupo. Utilizando estos datos analizaremos efectivamente cuales son las características mas predominantes y como repercute esas diferencias en el habla. Por último, mostraremos distintos clasificadores para determinar de que grupo proviene una grabación, analizaremos las atributos mas importantes y testearemos la solución propuesta. 

\bigskip
