% Un hablante para test, lo demás para train
\subsection{Un hablante para test y los demás para train}

\usetikzlibrary{shapes.geometric}

\tikzset{myshade/.style={minimum size=.4cm,shading=radial,inner color=white,outer color={#1!90!gray}}}
\newcommand\mycirc[1][]{\tikz\node[circle,myshade=#1]{};}

Tenemos 27 hablantes.

Vamos a separar uno para test y todos los demás para train. Cuando nos referimos a un hablante nos referimos a todos los audios grabados por él.

\begin{center}
	\mycirc[blue] Hablante para train \mycirc[red] Hablante para test
\end{center}

\begin{table}[H]
	\centering
	\begin{tabular}{cccccccccccc}
		& \multicolumn{11}{c}{\textit{Número de hablante}} \\
		& 1 & 2 & 3 & 4 & 5 & 6 & 7 & ... & 25 & 26 & 27 \\
		\hline \\
		Fold 1 &\mycirc[red] & \mycirc[blue] & \mycirc[blue]  & \mycirc[blue]  & \mycirc[blue]  & \mycirc[blue]  & \mycirc[blue] & ... & \mycirc[blue] & \mycirc[blue] & \mycirc[blue]  \\
		
		Fold 2 &\mycirc[blue] & \mycirc[red] & \mycirc[blue]  & \mycirc[blue]  & \mycirc[blue]  & \mycirc[blue]  & \mycirc[blue] & ... & \mycirc[blue] & \mycirc[blue] & \mycirc[blue]  \\
		
		Fold 3 &\mycirc[blue] & \mycirc[blue] & \mycirc[red]  & \mycirc[blue]  & \mycirc[blue]  & \mycirc[blue]  & \mycirc[blue] & ... & \mycirc[blue] & \mycirc[blue] & \mycirc[blue]  \\
	
		\multicolumn{11}{c}{\textit{...}}	\\
		
		Fold 27 &\mycirc[blue] & \mycirc[blue] & \mycirc[blue]  & \mycirc[blue]  & \mycirc[blue]  & \mycirc[blue]  & \mycirc[blue] & ... & \mycirc[blue] & \mycirc[blue] & \mycirc[red]   \\
	
	\end{tabular}
	\caption{Esquema de cross-validation}
	\label{}
\end{table}
		
\subsection{Resultados}

\begin{table}[H]
	\centering
	\begin{tabular}{|l|c|c|c|c|c|c|}
		\hline
		\textbf{}  & \textbf{ZeroR} & \textbf{JRip} & \textbf{J48} & \textbf{Function SMO} & \textbf{NaiveBayes} \\ \hline
		%\textbf{Fold 1}  &  &  &  &  &  \\ \hline
		%\hline \hline
		\textbf{Promedio} & 70.37  & 69.47 & 70.37 & 71.34 & 71.46 \\ \hline
	\end{tabular}
	\caption{Clasificación correcta en porcentaje}
	\label{}
\end{table}


\subsubsection{Wilcoxon y Test t de Student}

Valores del p-valor sobre cada test.

\begin{table}[H]
	\centering
	\begin{tabular}{|l|c|c|c|c|c|c|}
		\hline
		\textbf{}  & \textbf{Student Test} & \textbf{Wilcoxon Test} \\ \hline
		\textbf{ZeroR y JRip}  & 0.5816 & 0.6234 \\ \hline
		\textbf{ZeroR y J48}  & 1 & 1 \\ \hline
		\textbf{ZeroR y NaiveBayes}  & 0.4383 & 0.4042 \\ \hline
		\textbf{ZeroR y Function SMO}  & 0.4302 & 0.2646 \\ \hline
	\end{tabular}
	\caption{Resultados de cada test representado en p-valor}
	\label{res_tests_wilcoxon_student}
\end{table}

% hablar de los problemas encontrados y cómo mejorarlos con el próximo cross-validation 

Problemas:
\begin{itemize}
	\item desbalance
	\item missing values
\end{itemize}