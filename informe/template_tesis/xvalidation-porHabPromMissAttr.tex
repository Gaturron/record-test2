% Cada hablante que tenga un atributo '?' cambiarlo por el promedio de ese atributo en las demás audios
\subsection{Promediando los atributos de cada hablante sólo si es desconocido}

En el anterior esquema vimos que promediando los atributo evitamos tener valores desconocidos. Sí bien esto es cierto, no es necesario promediar en todo los casos. 

Veamos un ejemplo: supongamos que tenemos como conjunto de datos a la Tabla \ref{attr_orig}. Si para cada hablante promediamos sus atributos estaríamos perdiendo información. El hablante 1 tiene en el Audio1 el atributo A1 definido como 1, mientras que en el Audio3 como 2. Si realizaramos el promedio, estos valores los perderíamos.

\begin{table}[H]
	\centering
	\begin{tabular}{|l|l|ccccc|}
		\hline
		\multicolumn{2}{|l|}{Atributos} & A1 & A2 & A3 & ... & AN \\
		\hline 
		\textbf{Hablante 1} & \textbf{Audio1} & 1 & ? & 2 & & 2\\
		& \textbf{Audio2} & ? & ? & 1 & ... & ? \\
		& \textbf{Audio3} & 2 & ? & 3 & & ? \\
		\hline
		\textbf{Hablante 2} & \textbf{Audio1} & 1 & ? & ? & ... & ? \\
		& \textbf{Audio2} & 1 & 2 & ? & & ? \\
		\hline
	\end{tabular}
	\caption{Atributos original}
	\label{attr_orig}
\end{table}

Es por ello que proponemos esta variante. Cuando haya un atributo desconocido en un audio, vamos a promediarlo con los atributos de los demás audios del mismo hablante. En la tabla \ref{attr_mod} se puede ver el resultado de esta variante. 

Por ejemplo: para el hablante 1, el audio1 no tiene definido el atributo 1. Entonces vamos a promediarlo con los demás audios. El audio1 y audio3 sí tienen este atributo definido y sus valores son 2 y 1 respectivamente. Realizamos el promedio: $ 2 + 1 / 2 = 1,5$ entonces el valor del atributo 1 para el audio2 es 1,5. De esta forma, no perdemos información con respecto al audio extraído.

\begin{table}[H]
	\centering
	\begin{tabular}{|l|l|ccccc|}
		\hline
		\multicolumn{2}{|l|}{Atributos} & A1 & A2 & A3 & ... & AN \\
		\hline 
		\textbf{Hablante 1} & \textbf{Audio1} & 1 & ? & 2 & & 2\\
		& \textbf{Audio2} & \textbf{1.5} & ? & 1 & ... & \textbf{2} \\
		& \textbf{Audio3} & 2 & ? & 3 & & \textbf{2} \\
		\hline
		\textbf{Hablante 2} & \textbf{Audio1} & 1 & \textbf{2} & ? & ... & ? \\
		& \textbf{Audio2} & 1 & 2 & ? & & ? \\
		\hline
	\end{tabular}
	\caption{Atributos modificado}
	\label{attr_mod}
\end{table}

\begin{table}[H]
	\centering
	\begin{tabular}{|l|c|c|c|c|c|c|}
		\hline
		\textbf{}  & \textbf{ZeroR} & \textbf{JRip} & \textbf{J48} & \textbf{Function SMO} & \textbf{NaiveBayes} \\ \hline
		\textbf{Fold 1}  &  &  &  &  &  \\ \hline
		\hline \hline
		\textbf{Promedio} &  &  &  &  &  \\ \hline
	\end{tabular}
	\caption{Clasificación correcta en porcentaje}
	\label{class_corr_en_pct}
\end{table}

\subsubsection{Wilcoxon y Test t de Student}

\begin{table}[H]
	\centering
	\begin{tabular}{|l|c|c|c|c|c|c|}
		\hline
		\textbf{}  & \textbf{Student Test} & \textbf{Wilcoxon Test} \\ \hline
		\textbf{ZeroR y JRip}  &  &  \\ \hline
		\textbf{ZeroR y J48}  &  &  \\ \hline
		\textbf{ZeroR y NaiveBayes}  &  &  \\ \hline
		\textbf{ZeroR y Function SMO}  &  & \\ \hline
	\end{tabular}
	\caption{Resultados de cada test representado en p-valor}
	\label{res_tests_wilcoxon_student}
\end{table}

\subsection{Características del modelo de test}

Con esta esquema mejoramos las matrices de confusión. La siguiente corresponde al fold 1. Ahora se extrae mas información.	

\begin{table}[H]
	\centering
	\begin{tabular}{|c|c|c|}
		\hline
		Buenos Aires & Córdoba & \\ \hline
		33 & 1 & Buenos Aires\\ \hline
		0 & 0 & Córdoba\\ \hline
	\end{tabular}
\end{table}

% hablar de los problemas encontrados y cómo mejorarlos con el próximo cross-validation 