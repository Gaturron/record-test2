\chapter{Trabajos futuros}

Algunos trabajos futuros que se desprenden de este trabajo son:

%\paragraph*{Filtro por ruido con SoX:} Se podría analizar mejor los audios si se le aplicara algún filtro que extraiga el ruido. De esta forma se podría mejorar los atributos acústicos. 

%\paragraph*{Chequeador cruzado con grabación:} Que no se muestre la frase a decir, sino que se escuche un audio y se tenga repetirlo. Este audio es el de un hablante anterior pero que se le aplicó un filtro para evitar exponer su acento. De esta forma, chequeamos que se diga lo que se quiere decir y nos aseguramos espontaneidad.

\paragraph*{Chequeador cruzado:} Que el hablante diga si una grabación de un hablante anterior dijo lo que tenía que decir, o se equivocó. En este caso, serviría mucho para chequear los datos. 

Una mejora, por parte del análisis de datos, podría ser que los audios sean chequeados entre los hablantes. Un hablante, entre grabación y grabación, debe escuchar un audio (y su frase asociada) de otro hablante que previamente realizó el experimento. Luego de escucharla, deberá decidir si en la grabación se escucha correctamente la frase en cuestión. Si es así, se podrá ya decidir que esa grabación es buena para el extractor y que se mantiene como conservada.

Si se logra que cada hablante pueda chequear si otra frase se dijo correctamente permitiría que no sea necesario por parte de los administradores del sistema realizar este trabajo. Además balancea la carga 

\paragraph*{Validación de calidad de sonido:} En el momento de grabación analizar el audio grabado y rechazarlo si no supera un nivel aceptable auditivo. Esto puede implementarse de varias formas. Una posibilidad sería cuando esta grabando medir el volumen del micrófono cada una cierta cantidad de tiempo (por ejemplo: 1 segundo). Si en esa medición el volumen no se encuentra entre rango máximo y mínimo de volumen, descartar el audio y pedirle al hablante que vuelva a grabar.

También se le podría dar más información al hablante. Sabiendo que el micrófono tuvo un pico de volumen, se podría pedir al hablante que baje el nivel de voz o se aleje del micrófono. Ídem si habla muy bajo. Otras posibles soluciones a este problema son: analizar, antes de empezar el experimento, si el ruido ambiente no genera saturación. En tal caso, pedir al usuario que realice el experimento en un lugar más silencioso. 

Podemos realizar análisis más precisos sobre la calidad del audio cuando llega la grabación al servidor. En ese momento, el servidor ya puede obtener el archivo wav y realizarle todo tipo de análisis (por ejemplo: detección de ruido ambiente). Recordemos que el servidor esta implementado en Python que posee muchas librerías útiles para realizar esto. Al momento de terminar de procesar el audio en cuestión, deberá enviar la respuesta al hablante informándole si se debe realizar devuelta la grabación o si fue exitosa. Es importante notar que esta solución necesita buena conexión para servidor. 

\paragraph*{Clasificación en vivo:} Realizar la clasificación al agregar un hablante nuevo. De esta forma, se puede darle una respuesta a que grupo pertenece. Revalidar el modelo al agregarse un nuevo Hablante. 
