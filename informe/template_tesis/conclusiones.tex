\chapter{Conclusiones y Trabajo Futuro}

% explico problemas que tuvimos en general
En este trabajo pudimos desarrollar una herramienta para el estudio de las variantes argentinas del español. Diseñamos desde cero una página web que nos permitió llevar a cabo experimentos del habla. Pudimos analizar las diferencias de dos variantes importantes de la Argentina: una radicada en Buenos Aires y la otra en Córdoba. 

Al analizar las grabaciones recolectadas, nos enfrentamos a distintos problemas para poder utilizar esos datos. La elección de atributos a tener en cuenta y la precisa alineación de los mismos para cada grabación fueron actividades importantes que se reflejaron en los datos posteriores. Fue muy interesante desarrollar todos los pasos, desde la obtención de las grabaciones hasta el análisis de clasificación, para ver cómo influyen en el resultado final.

% explico el analisis
Una vez obtenidos los datos y extraídos los atributos, analizamos cómo testear debidamente los clasificadores entrenados. Pudimos definir un clasificador baseline que logramos superar proponiendo diferentes clasificadores. Encontramos que \textit{Function SMO} tuvo mejor performance ya que es un clasificador que busca el hiperplano de margen máximo entre los dos grupos, es decir, con esta división generaliza mejor que los otros clasificadores probados. También analizamos qué atributos tienen más información a la hora de clasificar en cada grupo y comprobamos que estos son los mismos que uno intuye popularmente.

Realizando este trabajo aprendimos la dificultad que surge al tener un dataset desequilibrado. Esto fue un inconveniente a superar y se notó al realizar el modelo de testing. Otra lección que aprendimos es que, si bien los atributos elegidos fueron correctos, creemos que se podrían mejorar los atributos acústicos; teniendo grabaciones más limpias y sin ruido, la calidad de estos sería mejor y por ende mejor la información que aportan. 

% explico los trabajos futuros
A partir de este trabajo pueden surgir muchas mejoras descriptas en los trabajos futuros, que enumeraremos a continuación:

\chapter{Trabajos futuros}

Algunos trabajos futuros que se desprenden de este trabajo son:

%\paragraph*{Filtro por ruido con SoX:} Se podría analizar mejor los audios si se le aplicara algún filtro que extraiga el ruido. De esta forma se podría mejorar los atributos acústicos. 

%\paragraph*{Chequeador cruzado con grabación:} Que no se muestre la frase a decir, sino que se escuche un audio y se tenga repetirlo. Este audio es el de un hablante anterior pero que se le aplicó un filtro para evitar exponer su acento. De esta forma, chequeamos que se diga lo que se quiere decir y nos aseguramos espontaneidad.

\paragraph*{Chequeador cruzado:} Que el hablante diga si una grabación de un hablante anterior dijo lo que tenía que decir, o se equivocó. En este caso, serviría mucho para chequear los datos. 

Una mejora, por parte del análisis de datos, podría ser que los audios sean chequeados entre los hablantes. Un hablante, entre grabación y grabación, debe escuchar un audio (y su frase asociada) de otro hablante que previamente realizó el experimento. Luego de escucharla, deberá decidir si en la grabación se escucha correctamente la frase en cuestión. Si es así, se podrá ya decidir que esa grabación es buena para el extractor y que se mantiene como conservada.

Si se logra que cada hablante pueda chequear si otra frase se dijo correctamente permitiría que no sea necesario por parte de los administradores del sistema realizar este trabajo. Además balancea la carga 

\paragraph*{Validación de calidad de sonido:} En el momento de grabación analizar el audio grabado y rechazarlo si no supera un nivel aceptable auditivo. Esto puede implementarse de varias formas. Una posibilidad sería cuando esta grabando medir el volumen del micrófono cada una cierta cantidad de tiempo (por ejemplo: 1 segundo). Si en esa medición el volumen no se encuentra entre rango máximo y mínimo de volumen, descartar el audio y pedirle al hablante que vuelva a grabar.

También se le podría dar más información al hablante. Sabiendo que el micrófono tuvo un pico de volumen, se podría pedir al hablante que baje el nivel de voz o se aleje del micrófono. Ídem si habla muy bajo. Otras posibles soluciones a este problema son: analizar, antes de empezar el experimento, si el ruido ambiente no genera saturación. En tal caso, pedir al usuario que realice el experimento en un lugar más silencioso. 

Podemos realizar análisis más precisos sobre la calidad del audio cuando llega la grabación al servidor. En ese momento, el servidor ya puede obtener el archivo wav y realizarle todo tipo de análisis (por ejemplo: detección de ruido ambiente). Recordemos que el servidor esta implementado en Python que posee muchas librerías útiles para realizar esto. Al momento de terminar de procesar el audio en cuestión, deberá enviar la respuesta al hablante informándole si se debe realizar devuelta la grabación o si fue exitosa. Es importante notar que esta solución necesita buena conexión para servidor. 

\paragraph*{Clasificación en vivo:} Realizar la clasificación al agregar un hablante nuevo. De esta forma, se puede darle una respuesta a que grupo pertenece. Revalidar el modelo al agregarse un nuevo Hablante. 
