\chapter{Conclusiones}

% explico problemas que tuvimos en general
En este trabajo pudimos desarrollar una herramienta para el estudio de las variantes argentinas del español. Diseñamos desde cero una página web que nos permitió llevar a cabo experimentos del habla. Pudimos analizar las diferencias  de las dos variantes más importantes de la Argentina: una radicada en Buenos Aires y la otra en Córdoba. 

Al analizar las grabaciones recolectadas, nos enfrentamos a distintos problemas para poder utilizar esos datos. La elección de atributos a tener en cuenta y la precisa alineación de los mismos para cada audio fueron actividades importantes que se reflejaron en los datos posteriores. Fue muy interesante analizar todos los pasos, desde la obtención del audio hasta el análisis de clasificación, para ver como influye en el resultado final.

% explico el analisis
Una vez obtenido los datos y extraemos cada uno de los atributos, analizamos cómo poder testear debidamente los clasificadores entrenados. Pudimos definir el baseline que logramos superar proponiendo diferentes clasificadores. También analizamos qué atributos tienen más información a la hora de clasificar en cada grupo y comprobamos que estos son los mismos que uno intuye popularmente.

% explico los trabajos futuros
Notamos a partir de este trabajo que pueden surgir muchas mejoras descriptas en los trabajos futuros. Estas atacan los problemas más importantes que tuvimos: tener mejor calidad para las grabaciones y automatizar lo más posible el análisis de los audios. También poder realizar una clasificación en tiempo real y realimentar el audio clasificado al conjunto de datos sería un paso importante para esta herramienta. 
